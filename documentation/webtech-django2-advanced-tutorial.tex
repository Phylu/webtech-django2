%\documentclass[a4paper,10pt]{scrreprt}
\documentclass[a4paper,10pt]{scrartcl}


%++++++++++++++++++++++++++++++++++++++++
%+ Includes that make LaTeX life easier +
%++++++++++++++++++++++++++++++++++++++++
%\usepackage[utf8x]{inputenc}			% Used for pdfLaTex
\usepackage{xltxtra}				% Use XeLaTeX
\usepackage{xunicode}				% Unicode input
\usepackage[english]{babel}			% German language
\usepackage{color}				% Define colors and use them
\usepackage{amssymb}				% Mathematical symbols
\usepackage{fontspec}
\usepackage{graphics}				% Needed to include pictures
\usepackage{float}				% Needed for picture placing
\usepackage{wrapfig}
\usepackage{amsmath}
\usepackage[hyphens]{url}
\usepackage[breaklinks=true]{hyperref}		% URLs
\usepackage[all]{hypcap}
\usepackage{mathcomp}
\usepackage{enumerate}				% More enumerate styles
\usepackage{listings}				% Listings (for code)
\usepackage{tikz}
\usepackage{geometry}
\usepackage{titling}
\usepackage{rotating}
\usepackage{bchart}

\usepackage{fourier}
\usepackage{lmodern}
\renewcommand*\familydefault{\sfdefault}
\usepackage[T1]{fontenc}

\usepackage{fullpage}
%\usepackage[all]{background}

\usepackage{booktabs} % Allows the use of \toprule, \midrule and \bottomrule in tables
\usepackage{wasysym} % CheckedBox and XBox
\usepackage{tabularx} % Tables with linebreaks
\usepackage{csvsimple} % Read csv files
\usepackage{longtable}
\usepackage{tabu}

\usepackage[width=3cm,heightr=1,filledcolor=next9red,emptycolor=next9green!30]{progressbar}
\usepackage{pdflscape}

\definecolor{dkgreen}{rgb}{0,.6,0}
\definecolor{dkblue}{rgb}{0,0,.6}
\definecolor{dkyellow}{cmyk}{0,0,.8,.3}

\definecolor{ocherCode}{rgb}{1, 0.5, 0}

%++++++++++++++++++++++++++++++++++++++++
%+   Write code in listings environment +
%++++++++++++++++++++++++++++++++++++++++
% Set lstlisting parameters
\lstset{ %
%language=Bash,                % the language of the code
basicstyle=\ttfamily\color{black},       % the size of the fonts that are used for the code
numbers=right,                   % where to put the line-numbers
%numberstyle=\footnotesize,      % the size of the fonts that are used for the line-numbers
%stepnumber=2,                   % the step between two line-numbers. If it's 1, each line 
                                % will be numbered
%numbersep=5pt,                  % how far the line-numbers are from the code
%backgroundcolor=\color{white},  % choose the background color. You must add \usepackage{color}
showspaces=false,               % show spaces adding particular underscores
showstringspaces=false,         % underline spaces within strings
showtabs=false,                 % show tabs within strings adding particular underscores
%frame=single,                   % adds a frame around the code
tabsize=2,                      % sets default tabsize to 2 spaces
captionpos=b,                   % sets the caption-position to bottom
breaklines=true,                % sets automatic line breaking
breakatwhitespace=false,        % sets if automatic breaks should only happen at whitespace
%title=\lstname,                 % show the filename of files included with \lstinputlisting;
                                % also try caption instead of title
%escapeinside={\%*}{*)},         % if you want to add a comment within your code
%morekeywords={*,...},           % if you want to add more keywords to the set
%keywordstyle=\color{blakc},     % color of keywords
%stringstyle=\color{black},    % color of strings
%identifierstyle=\color{black},    % color of variables
%frame=lines                     % Oberhalb und unterhalb des Listings ist eine Linie
} 
  
\lstdefinestyle{Bash} {
    language        = sh,
    basicstyle      = \small\ttfamily,
    keywordstyle    = \color{dkblue},
    stringstyle     = \color{red},
    identifierstyle = \color{dkgreen},
    commentstyle    = \color{gray}}

\lstdefinestyle{HTML} {
    language        = HTML,
    basicstyle      = \small\ttfamily,
  keywordstyle    = \color{dkblue},
  stringstyle     = \color{ocherCode},
  identifierstyle = \color{black},
  commentstyle    = \color{gray}}

\lstdefinestyle{Python} {
    language        = Python,
    basicstyle      = \small\ttfamily,
  keywordstyle    = \color{dkblue},
  stringstyle     = \color{red},
  identifierstyle = \color{dkgreen},
  commentstyle    = \color{gray}}

\lstdefinestyle{CSS} {
    language        = CSS,
    basicstyle      = \small\ttfamily,
  keywordstyle    = \color{dkblue},
  stringstyle     = \color{red},
  identifierstyle = \color{dkgreen},
  commentstyle    = \color{gray}}
  

\floatplacement{figure}{H}  % Pictures are forced to their position
\floatplacement{table}{H}

\newcommand{\myfigure}[3]{
\begin{figure}
        \centering
                \includegraphics[width=0.9\textwidth]{#1} %Argument 1: Filename
                \caption{#2} %Argument 2: Caption / Bildunterschrift
        \label{#3} %Argument 3: Bezeichner
\end{figure}
}

\newcommand{\myfigurescaled}[4]{
\begin{figure}
        \centering
                \includegraphics[width=#4\textwidth]{#1} %Argument 1: Filename
                \caption{#2} %Argument 2: Caption / Bildunterschrift
        \label{#3} %Argument 3: Bezeichner
\end{figure}
}

\title{Python and Django - The elephant in the room (Part 2)}
\author{Janosch Maier <maierj@in.tum.de>}
\date{April 09, 2015}

%++++++++++++++++++++++++++++++++++++++++
%+   Making URLs looking good           +
%++++++++++++++++++++++++++++++++++++++++
\urlstyle{same}					% All urls look the same
\definecolor{black}{rgb}{0,0,0}			% Define color black
\hypersetup{colorlinks=true, breaklinks=true, linkcolor=black, menucolor=black, urlcolor=black}	% Use plain black links everywhere


\begin{document}
\maketitle
\tableofcontents
\section{Introduction}

This tutorial let you build a scorecard application. Seminars and lectures can be added and then voted up or down. An overview page shows the courses ranked by the votes. This tutorial was build for the TUM seminar Webtech within the summer term 2015. Refer to \url{https://wwwmatthes.in.tum.de/pages/g78qcvnwz3u3/Web-Technologies-Frameworks-Libraries-and-Plattforms} for more information.

\subsection{Requirements}

To go through this tutorial you will need an installation of Python 3 with Django 1.7. You will probably get Python by your package manager. For help installing Django you should definitely check out \url{https://docs.djangoproject.com/en/1.7/intro/install/}. If you encounter any problems during this tutorial, refer back to that page. There is a lot of further information there.

\subsection{Setup}
You can either clone the application from GitHub or create it yourself.

\subsubsection{Clone}
To get the application from the GitHub use:
\begin{lstlisting}[style=Bash, caption=Clone application, label=lst:clone_app]
git clone https://github.com/Phylu/webtech-django2.git
\end{lstlisting}

\subsubsection{Create}
If you want to start from scratch, create your development directory and create the application yourself. The application will be stored within the directory scorecard.
\begin{lstlisting}[style=Bash, caption=Create application, label=lst:create_app]
django-admin startproject webtech
cd webtech
./manage.py startapp scorecard
./manage.py migrate
\end{lstlisting}

Edit the file \emph{webtech/settings.py} and add your newly created django app.

\begin{lstlisting}[style=Python, caption=Register application, label=lst:register_app]
INSTALLED_APPS = (
    'django.contrib.admin',
    'django.contrib.auth',
    'django.contrib.contenttypes',
    'django.contrib.sessions',
    'django.contrib.messages',
    'django.contrib.staticfiles',
    'scorecard',
)
\end{lstlisting}

You might need to create some directories and template files yourself if you do not use the template from GitHub. This is currently not supported and most likely will never be.

\subsection{Test the application}
To start the server run the following command. You can then access the django application in your browser via \url{http://127.0.0.1:8000/scorecard}
\begin{lstlisting}[style=Bash, caption=Run development server, label=lst:run_server]
./manage.py runserver
\end{lstlisting}
\section{Object-Relational Mapping}

Django stores the objects of its models in a relational database. To access these from within the programming environment Object-Relational Mapping (ORM) is used. Each model is represented by a table with the model's attributes as its columns. ORM allows the programmer to access the stored models in an easy way.

\subsection{Lecturer and Course Model}
The following models are used for the scoreboard application:

\begin{lstlisting}[style=Python, caption=scorecard/models.py, label=lst:models.py]
from django.db import models


class Lecturer(models.Model):
    """
    This class stores the information about one lecturer
    first_name  is the field which stores the first name
    last_name   is the field which stores the last name
    pk          is created automatically as the primary key
    """
    first_name = models.CharField(max_length=200)
    last_name = models.CharField(max_length=200)

    def __str__(self):
        """
        Convert the Lecturer Name to a human readable string
        :return: Returns the full name of the lecturer
        """
        return "{0} {1}".format(self.first_name, self.last_name)


class Course(models.Model):
    """
    This class stores the information about one course
    course_title    is the field which stores the title of a course
    vote            is the number of votes a course got
    lecturer        refers to the Lecturer model and identifies the lecturer responsible for the course
    pk              is created automatically as the primary key
    """
    course_title = models.CharField(max_length=200)
    votes = models.IntegerField(default=0)
    lecturer = models.ForeignKey(Lecturer)

    def __str__(self):
        """
        Convert a Course Model to a human readable string
        :return: Returns the title of the course
        """
        return self.course_title

    def is_great_course(self):
        """
        Figures out if a course is great or not.
        Great courses have more than 100 positive votes
        :return: True if votes > 100, Fales otherwise
        """
        return self.votes > 100
\end{lstlisting}

\subsection{Adding Objects}

To work with the models, objects need to be created. The command \lstinline|Lecturer()| respectively \lstinline|Course()| create a new object of the corresponding class. The \lstinline|save()| method stores the object permanently in the database.
\begin{lstlisting}[style=Python, caption=Adding Objects, label=lst:adding_objects]
from scorecard.models import Course, Lecturer

l = Lecturer(first_name="Alexander", last_name="Waldmann")
l.save()
c = Course(course_title="Webtech", votes=0, lecturer=l)
c.save()
\end{lstlisting}

Those (and the following) commands can be either run within your django project classes (perhaps the views or a test class as described in the next chapter) or from the interactive shell. To enter the shell run from the terminal:
\begin{lstlisting}[style=Bash, caption=Open django shell, label=lst:django_shell]
./manage.py shell
\end{lstlisting}

\subsection{Retrieving Objects}
The attribute \lstinline|pk| is the primary key of an object. If not specified otherwise, it is an integer, that is automatically incremented. To get the objects we just created, you can call:
\begin{lstlisting}[style=Python, caption=Retrieving objects by Primary Key, label=lst:retrieving_objects_pk]
l = Lecturer.objects.get(pk=1)
c = Course.objects.get(pk=1)
\end{lstlisting}

To get all objects of one model, you can use the \lstinline|all()| method. This may be used for iterating or showing a set of objects in a template. You should probably add some more objects in order to get meaningfull results.
\begin{lstlisting}[style=Python, caption=Retrieving all objects of a model, label=lst:retrieving_objects_all]
for course in Course.objects.all():
  print(course)
\end{lstlisting}

To order the objects shown, there is the \lstinline|order_by()| method available. Order by can take several attributes for ordering. By default the order is ascending. A negative sign in front of the attribute name means descending ordering. The following statements gets all Lecturers ordered first by their first name, then by their last name in descending ordering. The following commands are similar:
\begin{lstlisting}[style=Python, caption=Retrieving objects in order, label=lst:retrieving_objects_ordered]
Lecturer.objects.all().order_by('-first_name', '-last_name')
Lecturer.objects.order_by('-first_name', '-last_name')
\end{lstlisting}

To restrict the output of objects to a certain amount or range, one can access the output of an objects method like an array. To get the three courses with the most votes, run:
\begin{lstlisting}[style=Python, caption=Retrieving ranges of objects, label=lst:retrieving_objects_range]
Course.objects.order_by('-votes')[0:3]
\end{lstlisting}


\subsection{Filtering}

Most likely, you will not want to get all objects, that are stored but filter your objects by certain criteria. For filtering one can use the \lstinline|filter()| method. To filter an object by its attributes, suffixes like \lstinline|__contains|, \lstinline|__startswith|, \lstinline|__exact|, \lstinline|__gte| are used. The filter in line 1 will match the Webtech course, that was created at the beginning of this chapter. The filter in line 2 will not match the course (at least if you use a database that has case-sensitive comparisons; So when using sqlite as database backend it will match the course as well). The filter in line 3 will match again, as the letter \lstinline|i| means, that the comparison is case-insensitive. The filter in line 4 matches all courses whose vote counter is greater than 3. The method \lstinline|exclude()| works the same way as \lstinline|filter()|, with all but the matched objects returned.
\begin{lstlisting}[style=Python, caption=Filtering objects, label=lst:filtering_objects]
Course.objects.filter(course_title__exact='Webtech')
Course.objects.filter(course_title__contains='web')
Course.objects.filter(course_title__icontains='web')
Course.objects.filter(votes__gte=3)
\end{lstlisting}

Filter who take another attribute of the same object into account use the \lstinline|F()| method. To find all lecturers whose first name is the same as their last name, use:
\begin{lstlisting}[style=Python, caption=Filtering objects with attribute comparisons, label=lst:filtering_objects_attributes]
from django.db.models import F
Lecturer.objects.filter(first_name__exact=F('last_name'))
\end{lstlisting}

Several filter statements are connected with a logical and. To create more sophisticated queries for example with or statements or negations, one can use the \lstinline|Q()| method.

The filter statements in the 2nd and 3rd line yield the same results. They return all lecturers whose first name contains the string Alex \emph{and} the last name does not contain the string Matthes.

The statement in the 4th line however returns all lecturers whose first name contains Alex \emph{or} the last name contains Matthes.

The last statement filters all lecturers whose first name contains Alex \emph{or} the last name contains \emph{not} Matthes.

\begin{lstlisting}[style=Python, caption=Filtering objects with or statements and negations, label=lst:filtering_objects_or]
from django.db.models import Q
Lecturer.objects.filter(first_name__contains='Alex', last_name__contains='Matthes')
Lecturer.objects.filter(Q(first_name__contains='Alex'), Q(last_name__contains='Matthes'))
Lecturer.objects.filter(Q(first_name__contains='Alex') | Q(last_name__contains='Matthes'))
Lecturer.objects.filter(Q(first_name__contains='Alex') | ~Q(last_name__contains='Matthes'))
\end{lstlisting}

\subsection{Aggregation}

With a relational database backend it is not only possible to filter out elements but do calculations already in the database. The following command counts the number of courses that are stored:
\begin{lstlisting}[style=Python, caption=Counting objects, label=lst:counting_objects]
Course.objects.count()
\end{lstlisting}

Further aggregation needs the corresponding imports from the django model class.

The command in line 1 and 2 get the average, respectively maximum votes of the stored courses.

The filter in line 4 gives all courses, who are voted highest of all courses.
\begin{lstlisting}[style=Python, caption=Average and Maximum values of objects, label=lst:objects_aggregations]
from django.db.models import Avg, Max
Course.objects.aggregate(Avg('votes'))['votes__avg']
Course.objects.aggregate(Max('votes'))['votes__max']
Course.objects.filter(Q(votes__exact=Course.objects.aggregate(Max('votes'))['votes__max']))
\end{lstlisting}

\subsection{Annotation}

Temporary attributes that depend other objects are helpful in some cases. To find out which lecturer performs best, one can group the courses by lecturers and add the average votes of all his courses.
\begin{lstlisting}[style=Python, caption=Annotating objects, label=lst:objects_annotations]
Course.objects.values('lecturer').annotate(avg_votes=Avg('votes'))
\end{lstlisting}

\subsection{Raw SQL Queries}
Using the \lstinline|raw()| method it is also possible to write raw SQL statements. This gives the programmer much freedom, however unescaped SQL statements may open the application for SQL injection attacks. So use this feature wisely. I will only refer to the django manual at this point\footnote{\url{https://docs.djangoproject.com/en/1.7/topics/db/sql/}}.


\subsection{Examples}

Several of those ORM techniques are used to create the statistics page for the scoreboard application. The \lstinline|get_best_lecturer_with_mean()| function returns the lecturer object with the highest votes mean and the votes mean itself. The statistics view sets the context for the information to be shown in the template.
\begin{lstlisting}[style=Python, caption=Statistics view, label=lst:statistics_view]
def get_best_lecturer_with_mean():
    best_lecturer = None
    best_lecturer_mean = -sys.maxsize
    lecturer_votes = Course.objects.values('lecturer').annotate(avg_votes=Avg('votes'))
    for lecturer_vote in lecturer_votes:
        if lecturer_vote['avg_votes'] > best_lecturer_mean:
            best_lecturer = lecturer_vote['lecturer']
            best_lecturer_mean = lecturer_vote['avg_votes']
    best_lecturer = Lecturer.objects.get(pk=best_lecturer)
    return best_lecturer, best_lecturer_mean
    
def statistics(request):
    """
    Show statistics page
    :param request:
    :return:
    """
    best_lecturer, best_lecturer_mean = get_best_lecturer_with_mean()
    context = {
        'lecturer_count': Lecturer.objects.count(),
        'courses_count': Course.objects.count(),
        'courses_votes_mean': Course.objects.aggregate(Avg('votes'))['votes__avg'],
        'lecturer_best': best_lecturer,
        'lecturer_best_votes_mean': best_lecturer_mean,
    }
    return render(request, 'scorecard/statistics.html', context)
\end{lstlisting}
\section{Tests}

Django has a sophisticated testing framework. You can directly test your functions with unit tests and do also further tests that simulate requests. The file \emph{scorecard/tests.py} contains the tests for the scorecard application. Test-driven development and continuous integration are posible with django.

\subsection{Configure the Testing Environment}
The class ScoreboardTest contains the tests for the scoreboard application. When tests are run, the application creates an extra database for the tests. all data in your productive environment is not touched. The \lstinline|setUp()| method (line 14) creates objects for the test cases. It is run automatically by the testing environment before each test. The test methods will use the defined lecturers and courses to run their tests.

\begin{lstlisting}[style=Python, caption=Exceprt from scorecard/tests.py, label=lst:tests.py]
from django.test import TestCase
from django.core.urlresolvers import reverse

from scorecard.models import Course, Lecturer


class ScoreboardTest(TestCase):
    lecturer_1 = Lecturer()
    lecturer_2 = Lecturer()
    course_1 = Course()
    course_2 = Course()
    course_3 = Course()

    def setUp(self):
        """
        Create some courses & lecturers
        """
        self.lecturer_1 = Lecturer.objects.create(first_name="Janosch", last_name="Maier")
        self.lecturer_2 = Lecturer.objects.create(first_name="A", last_name="B")
        self.course_1 = Course.objects.create(course_title="EADS", votes=0, lecturer=self.lecturer_1)
        self.course_2 = Course.objects.create(course_title="ITSec", votes=0, lecturer=self.lecturer_2)
        self.course_3 = Course.objects.create(course_title="Webtech", votes=0, lecturer=self.lecturer_2)
\end{lstlisting}

The test functions use several other methods for convenience. Important is \lstinline|vote()| (line 10), which takes a course and either 1 or -1 and votes the course up and down. The function returns the votes before and after the vote. This is not done directly within the model but using the views. \lstinline|self.client.get()| runs a get request on a page of the project. \lstinline|reverse| does a lookup in the \emph{urls.py} file to get the correct URL. \lstinline|vote_again()| (line 24) calls the vote\_again page to reenable the voting if a user in this session has already voted. This functionality is covered in section \ref{ssec:session} of this tutorial.

\begin{lstlisting}[style=Python, caption=Exceprt from scorecard/tests.py, label=lst:tests.py]
    def delete_courses(self):
        """
        Delete all courses for test
        :return:
        """
        self.course_1.delete()
        self.course_2.delete()
        self.course_3.delete()

    def vote(self, course, up_down):
        """
        Vote for course with vote up_down
        :param course: Course object
        :param up_down: 1 or -1
        :return: old_votes, new_votes
        """
        response = self.client.get(reverse('scorecard:index'))
        old_votes = response.context['object_list'].get(pk=course.pk).votes
        self.client.get(reverse('scorecard:vote', kwargs={'pk': course.pk, 'vote': up_down}))
        response = self.client.get(reverse('scorecard:index'))
        new_votes = response.context['object_list'].get(pk=course.pk).votes
        return old_votes, new_votes

    def vote_again(self):
        """
        Reset has_already_voted hook
        :return:
        """
        self.client.get(reverse('scorecard:vote_again'))
\end{lstlisting}

\subsection{Unittests}
\lstinline|test_lecturer_str()| works directly on the model and checks if the \lstinline|__str__()| method of the lecturer model works correctly

\begin{lstlisting}[style=Python, caption=Exceprt from scorecard/tests.py, label=lst:tests.py]
    def test_lecturer_str(self):
        """
        Test if the __str__ method of lecturer works properly
        :return:
        """
        lecturer = Lecturer.objects.create(first_name='Janosch', last_name='Maier')
        self.assertEqual(lecturer.__str__(), 'Janosch Maier')
\end{lstlisting}


\lstinline|test_great_course_for_bad_course()| and \lstinline|test_great_course_for_great_course()| also work on the model directly. They use the function \lstinline|is_great_course()|, which is defined in the course model. Those are regular unit tests.

\begin{lstlisting}[style=Python, caption=Exceprt from scorecard/tests.py, label=lst:tests.py]
    def test_great_course_for_bad_course(self):
        """
        Return false if course has less then 100 votes
        :return:
        """
        course = Course.objects.create(course_title="EADS", votes=100, lecturer=self.lecturer_1)
        self.assertEqual(course.is_great_course(), False)

    def test_great_course_for_great_course(self):
        """
        Return false if course has less then 100 votes
        :return:
        """
        course = Course.objects.create(course_title="EADS", votes=101, lecturer=self.lecturer_1)
        self.assertEqual(course.is_great_course(), True)
\end{lstlisting}

\subsection{Testing Webserver Respnoses}
\lstinline|test_error_msg_if_no_course()| first deletes all courses and then checks if the response for a get request on the index page contains the error message, that there are no courses available. This check is done using the \lstinline|assertContains()| method.

\begin{lstlisting}[style=Python, caption=Exceprt from scorecard/tests.py, label=lst:tests.py]
    def test_error_msg_if_no_course(self):
        """
        If there is no Course existing, the Index should present an error msg
        """
        self.delete_courses()
        response = self.client.get(reverse('scorecard:index'))
        self.assertContains(response, 'No courses available')
\end{lstlisting}


\lstinline|test_if_courses_are_shown_without_voting()| checks if all courses created in the \lstinline|setUp()| method are given to the index template correctly. \lstinline|assertQuerysetEqual()| checks if the objext\_list in the context equals the one that contains all the defined courses for the test case.

\begin{lstlisting}[style=Python, caption=Exceprt from scorecard/tests.py, label=lst:tests.py]
    def test_if_courses_are_shown_without_voting(self):
        """
        If courses are created, they should be shown on the index page
        """
        response = self.client.get(reverse('scorecard:index'))
        self.assertQuerysetEqual(response.context['object_list'], ['<Course: EADS>', '<Course: ITSec>', '<Course: Webtech>'])
\end{lstlisting}

\lstinline|test_if_voting_redirect_is_correct()| looks if the status code of the response when the voting page is called is a correct redirect.

\begin{lstlisting}[style=Python, caption=Exceprt from scorecard/tests.py, label=lst:tests.py]
    def test_if_voting_redirect_is_correct(self):
        """
        If a vote is counted, the user is redirected
        """
        response = self.client.get(reverse('scorecard:vote', kwargs={'pk':1, 'vote':1}))
        self.assertEqual(response.status_code, 302)
\end{lstlisting}

The following 4 test methods test the voting functionality. A vote shall only be counted when the user in the session has not yet voted (again, see section \ref{ssec:session} for more details). They use the \lstinline|vote()| method to call the voting page and compare the votes before and afterwards.

\begin{lstlisting}[style=Python, caption=Exceprt from scorecard/tests.py, label=lst:tests.py]
    def test_if_voting_increased_counter(self):
        """
        If a positive vote is counted, the counter should increase
        """
        vote = 1
        old_votes, new_votes = self.vote(self.course_1, vote)
        self.assertEqual(old_votes + vote, new_votes)

    def test_if_voting_decreased_counter(self):
        """
        If a negative vote is counted, the counter should decrease
        """
        vote = -1
        old_votes, new_votes = self.vote(self.course_1, vote)
        self.assertEqual(old_votes + vote, new_votes)

    def test_if_voting_increased_counter_twice_without_reset(self):
        """
        If a positive vote is counted, the counter should increase
        """
        vote = 1
        old_votes, not_used = self.vote(self.course_1, vote)
        not_used, new_votes = self.vote(self.course_1, vote)
        self.assertEqual(old_votes + vote, new_votes)

    def test_if_voting_increased_counter_twice_with_reset(self):
        """
        If a positive vote is counted, the counter should increase
        """
        vote = 1
        old_votes, not_used = self.vote(self.course_1, vote)
        self.vote_again()
        not_used, new_votes = self.vote(self.course_1, vote)
        self.assertEqual(old_votes + (2 * vote), new_votes)
\end{lstlisting}

\lstinline|test_statistic()| checks if the statistic pages contains the correct values that should be calculated with the lecturer and course objects as created using \lstinline|setUp()|. lecturer\_1 has only one course that is voted with two positive votes. The other lecture has an average of zero votes. Therefore lecturer\_1 is the best lecturer with a votes mean of two.

\begin{lstlisting}[style=Python, caption=Exceprt from scorecard/tests.py, label=lst:tests.py]
    def test_statistic(self):
        """
        Statistic should show correct values
        :return:
        """
        vote = 1
        self.vote(self.course_1, vote)
        self.vote_again()
        self.vote(self.course_1, vote)
        response = self.client.get(reverse('scorecard:statistics'))
        self.assertEqual(response.context['lecturer_best'], self.lecturer_1)
        self.assertEqual(response.context['lecturer_best_votes_mean'], 2)
\end{lstlisting}

To run the tests, use the following command:
\begin{lstlisting}[style=Bash, caption=Run tests, label=lst:run_tests]
./manage.py tests
\end{lstlisting}
\section{Middleware}
\label{sec:middleware}

Middlewares hook in between the request and the view. This means you can set additional information in the view easily or retrieve information that was not explicitly provided by the user with the request. Two prominent examples of this are the sessions and the messages module of django.

\section{Messages}

With django you can easily create messages and propagate them in between your pages. The messages application is already installed, if you create a new project. You just need to put some few statements into your \emph{webtech/config.py} to make it work. The context processor messages (line 2) makes your messages available for the templates. The second statement (line 6--9) is used to make error messages work propperly with bootstrap. The css class therefore has to be set to danger, not error.
\begin{lstlisting}[style=Python, caption=webtech/config.py, label=lst:config.py1]
TEMPLATE_CONTEXT_PROCESSORS = (
    'django.contrib.messages.context_processors.messages',
    'django.contrib.auth.context_processors.auth'
)

from django.contrib.messages import constants as messages
MESSAGE_TAGS = {
    messages.ERROR: 'danger'
}
\end{lstlisting}

The \lstinline|message.add_message()| method in the \emph{views.py} file adds some messages that are dilivered to the templates.
\begin{lstlisting}[style=Python, caption=Add messages to views, label=lst:views_msg]
from django.contrib import messages

def vote(request, pk, vote):
    """
    You can either up- or down-vote a course
    The vote is saved in the model
    Then the user is redirected to the index page
    :param request: Request to work on
    :param pk:      Primary key of the Course
    :param vote:    1 or -1 depending on up/down-vote
    :return:        Redirect to Index view or 404 error
    """
    # If there is no course with the corresponding pk return an error
    course = get_object_or_404(Course, pk=pk)
    if updateVote(course):
        messages.add_message(request, messages.SUCCESS, "Vote Successful")
        return HttpResponseRedirect(reverse('scorecard:index'))
    else:
        # Vote invalid
        messages.add_message(request, messages.WARNING, "Vote Not Successful")
        return HttpResponseRedirect(reverse('scorecard:index'))
\end{lstlisting}

To show the messages on your views page, use the following code. This uses djangos message.tags attribute to get the alert class for bootstrap.
\begin{lstlisting}[style=HTML, caption=index.html, label=lst:index.html1]

    
        <div class="alert alert-{{ message.tags }}">{{ message }}</div>
    

\end{lstlisting}

In general, one could also create a message variable in the view context himself. But the messaging framework simplifies this task and provides easy functions for this purpose.

\subsection{Session}
\label{ssec:session}

With the session middleware, one can store additional information for the requests that are connected with a session. In this scoreboard each user (identified by its session) shall only vote once. Therefore we store a hook in the session, if the user has already voted. For testing purposes, we allow the user to reset the hook by calling a special page.

You need to add the request context\_processor in \emph{scoreboard/config.py}, so you can access the request.session variable in the view/template.
\begin{lstlisting}[style=Python, caption=webtech/config.py, label=lst:config.py1]
TEMPLATE_CONTEXT_PROCESSORS = (
    'django.contrib.messages.context_processors.messages',
    'django.contrib.auth.context_processors.auth',
    'django.core.context_processors.request'
)
\end{lstlisting}

Update your \emph{views.py} in the following way to add the session storage handling. With \lstinline|request.session.get('has_voted', False)| (line 12) we try to retrieve the value of the has\_voted session variable. If there is nothing stored, the function returns \lstinline|False|. To store True in the variable you can simply use \lstinline|request.session['has_voted'] = True| (line 16).
\begin{lstlisting}[style=Python, caption=Session handling in vote view, label=lst:views_session]
def vote(request, pk, vote)
    """
    You can either up- or down-vote a course
    The vote is saved in the model
    Then the user is redirected to the index page
    :param request: Request to work on
    :param pk:      Primary key of the Course
    :param vote:    1 or -1 depending on up/down-vote
    :return:        Redirect to Index view or 404 error
    """
    # If there is no course with the corresponding pk return an error
    course = get_object_or_404(Course, pk=pk)
    if request.session.get('has_voted', False):
        messages.add_message(request, messages.ERROR, "You have already voted!")
        return index()
    if updateVote(course, vote):
        request.session['has_voted'] = True
        messages.add_message(request, messages.SUCCESS, "Vote Successful!")
        return index()
    else:
        # Vote invalid
        messages.add_message(request, messages.ERROR, "Vote Not Successful!")
        return index()

def vote_again(request):
    request.session['has_voted'] = False
    return index()
\end{lstlisting}

As we want the user to reset its session, so he can vote again, we show in information message if the has\_voted variable is set. The link should bring the user to the vote\_again page that resets the session hook (line 25 in listing \ref{lst:views_session})
\begin{lstlisting}[style=HTML, caption=index.html, label=lst:index.html1]
    
        <div class="alert alert-info">
        You usually can only vote once. If you want to vote again, I can make an exception for you. <a href="">Vote Again?</a>
        </div>
    
\end{lstlisting}

To access make the vote\_again page work, add a new URL redirect (line 4).
\begin{lstlisting}[style=Python, caption=urls.py, label=lst:urls.py1]
urlpatterns = patterns('',
                       url(r'^$', views.IndexView.as_view(), name='index'),
                       url(r'^(?P<pk>\d+)/(?P<vote>-?\d)/$', views.vote, name='vote'),
                       url(r'^vote_again$', views.vote_again, name='vote_again')
                       )
\end{lstlisting}
\section{Translation}

To make your django application available for a broad audience, internationalization is needed. With the translation framework of django this is possible.

\subsection{Translation definitions}
To create an internationalized version of the scorecard application, all strings need to put into translation functions. Based on these translation strings, the \emph{django-admin.py} file creates a \emph{.po} file, in which the translation takes place. First in your views (or wherever strings are created), import the translation method enclose the strings accordingly. Look at the string in line 6.

\begin{lstlisting}[style=Python, caption=exceprt of views.py with translation, label=lst:views.py_translation]
from django.utils.translation import ugettext as _

def vote(request, pk, vote):
    course = get_object_or_404(Course, pk=pk)
    if has_already_voted(request):
        messages.add_message(request, messages.ERROR, _("You have already voted!"))
\end{lstlisting}

For translating templates, the \lstinline|| tag is available. See how translation for the menu links is initialized in lines 5 and 6 of the following snippet.

\begin{lstlisting}[style=HTML, caption=exceprt of base.html with translation, label=lst:base.html_translation]


<div class="navbar-collapse" id="bs-example-navbar-collapse-1">
    <ul class="nav navbar-nav">
        <li><a href=""></a></li>
        <li><a href=""></a></li>
    </ul>
</div><!-- /.navbar-collapse -->
\end{lstlisting}

\subsection{Pluralization}
Pluralization is a bit more difficult. If English is set as the first language, the plural ``s'' can be used easily with the \lstinline|pluralize| command. This is not possible for more complicated languages such as German or any than the first language used in the project. To create a pluralized version of a string, you need a counter that defines whether the singular or plural version of the string is used. In the following code, \lstinline|course.votes| is used as counter, to pluralize the votes.

\begin{lstlisting}[style=HTML, caption=exceprt of index.html with translation, label=lst:index.html_translation]
{{ counter }} vote.{{ counter }} votes.
\end{lstlisting}

\subsection{Translation Files}
Now we defined the strings that shall be translated. To do the actual translation, let django create a translation file. Create the folder \emph{scorecard/locale}.

From within the \emph{scorecard} directory run the following command to create a translation file for German:

\begin{lstlisting}[style=Bash, caption=Create a German translation file, label=lst:makemessages]
django-admin.py makemessages -l de
\end{lstlisting}

Now in the directory \emph{scorecard/locale}, a directory for the German language appears. Deep down in this directory there is a \emph{django.po} file. Create the translation strings within this file. You can see the plural version of the string for the votes counter in the \emph{index.html} file in line 9-14.

\begin{lstlisting}[style=HTML, caption=django.po file for German, label=lst:django.po]
#: templates/scorecard/base.html:20
msgid "Home"
msgstr "Start"

#: templates/scorecard/base.html:21
msgid "Statistics"
msgstr "Statistik"

#: templates/scorecard/index.html:10
#, python-format
msgid "%(counter)s vote."
msgid_plural "%(counter)s votes."
msgstr[0] "%(counter)s Stimme."
msgstr[1] "%(counter)s Stimmen."
\end{lstlisting}

To apply the translated messages, run:
\begin{lstlisting}[style=Bash, caption=Compile messages, label=lst:compilemessages]
django-admin.py compilemessages
\end{lstlisting}
\end{document}
