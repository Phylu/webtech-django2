\section{Translation}

To make your django application available for a broad audience, internationalization is needed. With the translation framework of django this is possible.

\subsection{Translation definitions}
To create an internationalized version of the scorecard application, all strings need to put into translation functions. Based on these translation strings, the \emph{django-admin.py} file creates a \emph{.po} file, in which the translation takes place. First in your views (or wherever strings are created), import the translation method enclose the strings accordingly. Look at the string in line 6.

\begin{lstlisting}[style=Python, caption=exceprt of views.py with translation, label=lst:views.py_translation]
from django.utils.translation import ugettext as _

def vote(request, pk, vote):
    course = get_object_or_404(Course, pk=pk)
    if has_already_voted(request):
        messages.add_message(request, messages.ERROR, _("You have already voted!"))
\end{lstlisting}

For translating templates, the \lstinline|| tag is available. See how translation for the menu links is initialized in lines 5 and 6 of the following snippet.

\begin{lstlisting}[style=HTML, caption=exceprt of base.html with translation, label=lst:base.html_translation]


<div class="navbar-collapse" id="bs-example-navbar-collapse-1">
    <ul class="nav navbar-nav">
        <li><a href=""></a></li>
        <li><a href=""></a></li>
    </ul>
</div><!-- /.navbar-collapse -->
\end{lstlisting}

\subsection{Pluralization}
Pluralization is a bit more difficult. If English is set as the first language, the plural ``s'' can be used easily with the \lstinline|pluralize| command. This is not possible for more complicated languages such as German or any than the first language used in the project. To create a pluralized version of a string, you need a counter that defines whether the singular or plural version of the string is used. In the following code, \lstinline|course.votes| is used as counter, to pluralize the votes.

\begin{lstlisting}[style=HTML, caption=exceprt of index.html with translation, label=lst:index.html_translation]
{{ counter }} vote.{{ counter }} votes.
\end{lstlisting}

\subsection{Translation Files}
Now we defined the strings that shall be translated. To do the actual translation, let django create a translation file. Create the folder \emph{scorecard/locale}.

From within the \emph{scorecard} directory run the following command to create a translation file for German:

\begin{lstlisting}[style=Bash, caption=Create a German translation file, label=lst:makemessages]
django-admin.py makemessages -l de
\end{lstlisting}

Now in the directory \emph{scorecard/locale}, a directory for the German language appears. Deep down in this directory there is a \emph{django.po} file. Create the translation strings within this file. You can see the plural version of the string for the votes counter in the \emph{index.html} file in line 9-14.

\begin{lstlisting}[style=HTML, caption=django.po file for German, label=lst:django.po]
#: templates/scorecard/base.html:20
msgid "Home"
msgstr "Start"

#: templates/scorecard/base.html:21
msgid "Statistics"
msgstr "Statistik"

#: templates/scorecard/index.html:10
#, python-format
msgid "%(counter)s vote."
msgid_plural "%(counter)s votes."
msgstr[0] "%(counter)s Stimme."
msgstr[1] "%(counter)s Stimmen."
\end{lstlisting}

To apply the translated messages, run:
\begin{lstlisting}[style=Bash, caption=Compile messages, label=lst:compilemessages]
django-admin.py compilemessages
\end{lstlisting}