%\documentclass[a4paper,10pt]{scrreprt}
\documentclass[a4paper,10pt]{scrartcl}


%++++++++++++++++++++++++++++++++++++++++
%+ Includes that make LaTeX life easier +
%++++++++++++++++++++++++++++++++++++++++
%\usepackage[utf8x]{inputenc}			% Used for pdfLaTex
\usepackage{xltxtra}				% Use XeLaTeX
\usepackage{xunicode}				% Unicode input
\usepackage[english]{babel}			% German language
\usepackage{color}				% Define colors and use them
\usepackage{amssymb}				% Mathematical symbols
\usepackage{fontspec}
\usepackage{graphics}				% Needed to include pictures
\usepackage{float}				% Needed for picture placing
\usepackage{wrapfig}
\usepackage{amsmath}
\usepackage[hyphens]{url}
\usepackage[breaklinks=true]{hyperref}		% URLs
\usepackage[all]{hypcap}
\usepackage{mathcomp}
\usepackage{enumerate}				% More enumerate styles
\usepackage{listings}				% Listings (for code)
\usepackage{tikz}
\usepackage{geometry}
\usepackage{titling}
\usepackage{rotating}
\usepackage{bchart}

\usepackage{fourier}
\usepackage{lmodern}
\renewcommand*\familydefault{\sfdefault}
\usepackage[T1]{fontenc}

\usepackage{fullpage}
%\usepackage[all]{background}

\usepackage{booktabs} % Allows the use of \toprule, \midrule and \bottomrule in tables
\usepackage{wasysym} % CheckedBox and XBox
\usepackage{tabularx} % Tables with linebreaks
\usepackage{csvsimple} % Read csv files
\usepackage{longtable}
\usepackage{tabu}

\usepackage[width=3cm,heightr=1,filledcolor=next9red,emptycolor=next9green!30]{progressbar}
\usepackage{pdflscape}

\definecolor{dkgreen}{rgb}{0,.6,0}
\definecolor{dkblue}{rgb}{0,0,.6}
\definecolor{dkyellow}{cmyk}{0,0,.8,.3}

\definecolor{ocherCode}{rgb}{1, 0.5, 0}

%++++++++++++++++++++++++++++++++++++++++
%+   Write code in listings environment +
%++++++++++++++++++++++++++++++++++++++++
% Set lstlisting parameters
\lstset{ %
%language=Bash,                % the language of the code
basicstyle=\ttfamily\color{black},       % the size of the fonts that are used for the code
%numbers=left,                   % where to put the line-numbers
%numberstyle=\footnotesize,      % the size of the fonts that are used for the line-numbers
%stepnumber=2,                   % the step between two line-numbers. If it's 1, each line 
                                % will be numbered
%numbersep=5pt,                  % how far the line-numbers are from the code
%backgroundcolor=\color{white},  % choose the background color. You must add \usepackage{color}
showspaces=false,               % show spaces adding particular underscores
showstringspaces=false,         % underline spaces within strings
showtabs=false,                 % show tabs within strings adding particular underscores
%frame=single,                   % adds a frame around the code
tabsize=2,                      % sets default tabsize to 2 spaces
captionpos=b,                   % sets the caption-position to bottom
breaklines=true,                % sets automatic line breaking
breakatwhitespace=false,        % sets if automatic breaks should only happen at whitespace
%title=\lstname,                 % show the filename of files included with \lstinputlisting;
                                % also try caption instead of title
%escapeinside={\%*}{*)},         % if you want to add a comment within your code
%morekeywords={*,...},           % if you want to add more keywords to the set
%keywordstyle=\color{blakc},     % color of keywords
%stringstyle=\color{black},    % color of strings
%identifierstyle=\color{black},    % color of variables
%frame=lines                     % Oberhalb und unterhalb des Listings ist eine Linie
} 
  
\lstdefinestyle{Bash} {
    language        = sh,
    basicstyle      = \small\ttfamily,
    keywordstyle    = \color{dkblue},
    stringstyle     = \color{red},
    identifierstyle = \color{dkgreen},
    commentstyle    = \color{gray}}

\lstdefinestyle{HTML} {
    language        = HTML,
    basicstyle      = \small\ttfamily,
  keywordstyle    = \color{dkblue},
  stringstyle     = \color{ocherCode},
  identifierstyle = \color{black},
  commentstyle    = \color{gray}}

\lstdefinestyle{Python} {
    language        = PHP,
    basicstyle      = \small\ttfamily,
  keywordstyle    = \color{dkblue},
  stringstyle     = \color{red},
  identifierstyle = \color{dkgreen},
  commentstyle    = \color{gray}}

\lstdefinestyle{CSS} {
    language        = CSS,
    basicstyle      = \small\ttfamily,
  keywordstyle    = \color{dkblue},
  stringstyle     = \color{red},
  identifierstyle = \color{dkgreen},
  commentstyle    = \color{gray}}
  

\floatplacement{figure}{H}  % Pictures are forced to their position
\floatplacement{table}{H}

\newcommand{\myfigure}[3]{
\begin{figure}
        \centering
                \includegraphics[width=0.9\textwidth]{#1} %Argument 1: Filename
                \caption{#2} %Argument 2: Caption / Bildunterschrift
        \label{#3} %Argument 3: Bezeichner
\end{figure}
}

\newcommand{\myfigurescaled}[4]{
\begin{figure}
        \centering
                \includegraphics[width=#4\textwidth]{#1} %Argument 1: Filename
                \caption{#2} %Argument 2: Caption / Bildunterschrift
        \label{#3} %Argument 3: Bezeichner
\end{figure}
}

\title{Python and Django - The elephant in the room (Part 2)}
\author{Janosch Maier}
\date{April 09, 2015}

%++++++++++++++++++++++++++++++++++++++++
%+   Making URLs looking good           +
%++++++++++++++++++++++++++++++++++++++++
\urlstyle{same}					% All urls look the same
\definecolor{black}{rgb}{0,0,0}			% Define color black
\hypersetup{colorlinks=true, breaklinks=true, linkcolor=black, menucolor=black, urlcolor=black}	% Use plain black links everywhere


\begin{document}
\maketitle
\tableofcontents
\section{Introduction}

This tutorial let you build a scorecard application. Seminars and lectures can be added and then voted up or down. An overview page shows the courses ranked by the votes. This tutorial was build for the TUM seminar Webtech within the summer term 2015. Refer to \url{https://wwwmatthes.in.tum.de/pages/g78qcvnwz3u3/Web-Technologies-Frameworks-Libraries-and-Plattforms} for more information.

\subsection{Requirements}

To go through this tutorial you will need an installation of Python 3 with Django 1.7. You will probably get Python by your package manager. For help installing Django you should definitely check out \url{https://docs.djangoproject.com/en/1.7/intro/install/}. If you encounter any problems during this tutorial, refer back to that page. There is a lot of further information there.

\subsection{Setup}
You can either clone the application from GitHub or create it yourself.

\subsubsection{Clone}
To get the application from the GitHub use:
\begin{lstlisting}[style=Bash, caption=Clone application, label=lst:clone_app]
git clone https://github.com/Phylu/webtech-django2.git
\end{lstlisting}

\subsubsection{Create}
If you want to start from scratch, create your development directory and create the application yourself. The application will be stored within the directory scorecard.
\begin{lstlisting}[style=Bash, caption=Create application, label=lst:create_app]
django-admin startproject webtech
cd webtech
./manage.py startapp scorecard
./manage.py migrate
\end{lstlisting}

Edit the file \emph{webtech/settings.py} and add your newly created django app.

\begin{lstlisting}[style=Python, caption=Register application, label=lst:register_app]
INSTALLED_APPS = (
    'django.contrib.admin',
    'django.contrib.auth',
    'django.contrib.contenttypes',
    'django.contrib.sessions',
    'django.contrib.messages',
    'django.contrib.staticfiles',
    'scorecard',
)
\end{lstlisting}

You might need to create some directories and template files yourself if you do not use the template from GitHub. This is currently not supported and most likely will never be.

\subsection{Test the application}
To start the server run the following command. You can then access the django application in your browser via \url{http://127.0.0.1:8000/scorecard}
\begin{lstlisting}[style=Bash, caption=Run development server, label=lst:run_server]
./manage.py runserver
\end{lstlisting}
\section{Models}

\subsection{Course model}
Within models information is stored for the web application. We use one model witch stores a course and the number of votes it got. A positive vote increases the vote counter. A negative vote decreases it. Create the file \emph{scorecard/course.py} and fill it with the following content:

\begin{lstlisting}[style=Python, caption=scorecard/models.py, label=lst:models.py]
from django.db import models


class Course(models.Model):
    """
    This class stores the information about one course
    course_title    is the field which stores the title of a course
    vote            is the number of votes a course got
    pk              is created automatically as the primary key
    """
    course_title = models.CharField(max_length=200)
    votes = models.IntegerField(default=0)
\end{lstlisting}

As you can see, there is one character field with maximum length of 200 characters, that stores teh title of a course. Additionally there is a field votes that stores the number of votes one course got. The number of votes can go negative. Now the model has to be registered with our django installation. Run:
\begin{lstlisting}[style=Bash, caption=Register models, label=lst:register_models]
./manage.py makemigrations scorecard
./manage.py migrate
\end{lstlisting}

\subsection{Admin app for editing models}
If you try to access your django application (\url{http://127.0.0.1:8000/scorecard}, you remember?), you will get an error message. We have not yet defined the framework to show anything. To see if your model was propperly created and interact with it, django provides an admin interface in its package django.contrib.admin. This one is activated by default if you create a new django application.

To be shown on the admin interface, the new model needs to be enabled. You can do this by adding the model to the file \emph{scorecard/admin.py}.
\begin{lstlisting}[style=Python, caption=scorecard/admin.py, label=lst:admin.py]
from django.contrib import admin
from scorecard.models import Course

admin.site.register(Course)
\end{lstlisting}


To access it, you need to create a user first (In the version from github, the credentials are admin:admin). Run:
\begin{lstlisting}[style=Bash, caption=Create superuser, label=lst:create_superuser]
./manage.py createsuperuser
\end{lstlisting}
Now you can access \url{http://127.0.0.1:8000/admin} and create some courses as you like. As you can see, the course overview is not named nicely. This is because django does not know how to convert the course objects to a string. Add a \lstinline|__str__| method to the model to solve this. We let the string conversion return the title of the course, as this is human readable and for our case identifies the courses.

\begin{lstlisting}[style=Python, caption=scorecard/models.py, label=lst:models.py1]
from django.db import models


class Course(models.Model):
    """
    This class stores the information about one course
    course_title    is the field which stores the title of a course
    vote            is the number of votes a course got
    pk              is created automatically as the primary key
    """
    course_title = models.CharField(max_length=200)
    votes = models.IntegerField(default=0)

    def __str__(self):
        """
        Convert a Course Model to a human readable string
        :return: Returns the title of the course
        """
        return self.course_title
\end{lstlisting}

The admin application can be highly modified to match your needs concerning ordering or search of model content. However this is not covered in this tutorial. Please refer to the official documentation for more information.
\end{document}
