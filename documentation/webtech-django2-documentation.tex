%\documentclass[a4paper,10pt]{scrreprt}
\documentclass[a4paper,10pt]{scrartcl}


%++++++++++++++++++++++++++++++++++++++++
%+ Includes that make LaTeX life easier +
%++++++++++++++++++++++++++++++++++++++++
%\usepackage[utf8x]{inputenc}			% Used for pdfLaTex
\usepackage{xltxtra}				% Use XeLaTeX
\usepackage{xunicode}				% Unicode input
\usepackage[english]{babel}			% German language
\usepackage{color}				% Define colors and use them
\usepackage{amssymb}				% Mathematical symbols
\usepackage{fontspec}
\usepackage{graphics}				% Needed to include pictures
\usepackage{float}				% Needed for picture placing
\usepackage{wrapfig}
\usepackage{amsmath}
\usepackage[hyphens]{url}
\usepackage[breaklinks=true]{hyperref}		% URLs
\usepackage[all]{hypcap}
\usepackage{mathcomp}
\usepackage{enumerate}				% More enumerate styles
\usepackage{listings}				% Listings (for code)
\usepackage{tikz}
\usepackage{geometry}
\usepackage{titling}
\usepackage{rotating}
\usepackage{bchart}

\usepackage{fourier}
\usepackage{lmodern}
\renewcommand*\familydefault{\sfdefault}
\usepackage[T1]{fontenc}

\usepackage{fullpage}
%\usepackage[all]{background}

\usepackage{booktabs} % Allows the use of \toprule, \midrule and \bottomrule in tables
\usepackage{wasysym} % CheckedBox and XBox
\usepackage{tabularx} % Tables with linebreaks
\usepackage{csvsimple} % Read csv files
\usepackage{longtable}
\usepackage{tabu}

\usepackage[width=3cm,heightr=1,filledcolor=next9red,emptycolor=next9green!30]{progressbar}
\usepackage{pdflscape}

\definecolor{dkgreen}{rgb}{0,.6,0}
\definecolor{dkblue}{rgb}{0,0,.6}
\definecolor{dkyellow}{cmyk}{0,0,.8,.3}

\definecolor{ocherCode}{rgb}{1, 0.5, 0}

%++++++++++++++++++++++++++++++++++++++++
%+   Write code in listings environment +
%++++++++++++++++++++++++++++++++++++++++
% Set lstlisting parameters
\lstset{ %
%language=Bash,                % the language of the code
basicstyle=\ttfamily\color{black},       % the size of the fonts that are used for the code
%numbers=left,                   % where to put the line-numbers
%numberstyle=\footnotesize,      % the size of the fonts that are used for the line-numbers
%stepnumber=2,                   % the step between two line-numbers. If it's 1, each line 
                                % will be numbered
%numbersep=5pt,                  % how far the line-numbers are from the code
%backgroundcolor=\color{white},  % choose the background color. You must add \usepackage{color}
showspaces=false,               % show spaces adding particular underscores
showstringspaces=false,         % underline spaces within strings
showtabs=false,                 % show tabs within strings adding particular underscores
%frame=single,                   % adds a frame around the code
tabsize=2,                      % sets default tabsize to 2 spaces
captionpos=b,                   % sets the caption-position to bottom
breaklines=true,                % sets automatic line breaking
breakatwhitespace=false,        % sets if automatic breaks should only happen at whitespace
%title=\lstname,                 % show the filename of files included with \lstinputlisting;
                                % also try caption instead of title
%escapeinside={\%*}{*)},         % if you want to add a comment within your code
%morekeywords={*,...},           % if you want to add more keywords to the set
%keywordstyle=\color{blakc},     % color of keywords
%stringstyle=\color{black},    % color of strings
%identifierstyle=\color{black},    % color of variables
%frame=lines                     % Oberhalb und unterhalb des Listings ist eine Linie
} 
  
\lstdefinestyle{Bash} {
    language        = sh,
    basicstyle      = \small\ttfamily,
    keywordstyle    = \color{dkblue},
    stringstyle     = \color{red},
    identifierstyle = \color{dkgreen},
    commentstyle    = \color{gray}}

\lstdefinestyle{HTML} {
    language        = HTML,
    basicstyle      = \small\ttfamily,
  keywordstyle    = \color{dkblue},
  stringstyle     = \color{ocherCode},
  identifierstyle = \color{black},
  commentstyle    = \color{gray}}

\lstdefinestyle{Python} {
    language        = Python,
    basicstyle      = \small\ttfamily,
  keywordstyle    = \color{dkblue},
  stringstyle     = \color{red},
  identifierstyle = \color{dkgreen},
  commentstyle    = \color{gray}}

\lstdefinestyle{CSS} {
    language        = CSS,
    basicstyle      = \small\ttfamily,
  keywordstyle    = \color{dkblue},
  stringstyle     = \color{red},
  identifierstyle = \color{dkgreen},
  commentstyle    = \color{gray}}
  

\floatplacement{figure}{H}  % Pictures are forced to their position
\floatplacement{table}{H}

\newcommand{\myfigure}[3]{
\begin{figure}
        \centering
                \includegraphics[width=0.9\textwidth]{#1} %Argument 1: Filename
                \caption{#2} %Argument 2: Caption / Bildunterschrift
        \label{#3} %Argument 3: Bezeichner
\end{figure}
}

\newcommand{\myfigurescaled}[4]{
\begin{figure}
        \centering
                \includegraphics[width=#4\textwidth]{#1} %Argument 1: Filename
                \caption{#2} %Argument 2: Caption / Bildunterschrift
        \label{#3} %Argument 3: Bezeichner
\end{figure}
}

\title{Python and Django - The elephant in the room (Part 2)}
\author{Janosch Maier <maierj@in.tum.de>}
\date{April 09, 2015}

%++++++++++++++++++++++++++++++++++++++++
%+   Making URLs looking good           +
%++++++++++++++++++++++++++++++++++++++++
\urlstyle{same}					% All urls look the same
\definecolor{black}{rgb}{0,0,0}			% Define color black
\hypersetup{colorlinks=true, breaklinks=true, linkcolor=black, menucolor=black, urlcolor=black}	% Use plain black links everywhere


\begin{document}
\maketitle
\tableofcontents
\section{Introduction}

This tutorial let you build a scorecard application. Seminars and lectures can be added and then voted up or down. An overview page shows the courses ranked by the votes. This tutorial was build for the TUM seminar Webtech within the summer term 2015. Refer to \url{https://wwwmatthes.in.tum.de/pages/g78qcvnwz3u3/Web-Technologies-Frameworks-Libraries-and-Plattforms} for more information.

\subsection{Requirements}

To go through this tutorial you will need an installation of Python 3 with Django 1.7. You will probably get Python by your package manager. For help installing Django you should definitely check out \url{https://docs.djangoproject.com/en/1.7/intro/install/}. If you encounter any problems during this tutorial, refer back to that page. There is a lot of further information there.

\subsection{Setup}
You can either clone the application from GitHub or create it yourself.

\subsubsection{Clone}
To get the application from the GitHub use:
\begin{lstlisting}[style=Bash, caption=Clone application, label=lst:clone_app]
git clone https://github.com/Phylu/webtech-django2.git
\end{lstlisting}

\subsubsection{Create}
If you want to start from scratch, create your development directory and create the application yourself. The application will be stored within the directory scorecard.
\begin{lstlisting}[style=Bash, caption=Create application, label=lst:create_app]
django-admin startproject webtech
cd webtech
./manage.py startapp scorecard
./manage.py migrate
\end{lstlisting}

Edit the file \emph{webtech/settings.py} and add your newly created django app.

\begin{lstlisting}[style=Python, caption=Register application, label=lst:register_app]
INSTALLED_APPS = (
    'django.contrib.admin',
    'django.contrib.auth',
    'django.contrib.contenttypes',
    'django.contrib.sessions',
    'django.contrib.messages',
    'django.contrib.staticfiles',
    'scorecard',
)
\end{lstlisting}

You might need to create some directories and template files yourself if you do not use the template from GitHub. This is currently not supported and most likely will never be.

\subsection{Test the application}
To start the server run the following command. You can then access the django application in your browser via \url{http://127.0.0.1:8000/scorecard}
\begin{lstlisting}[style=Bash, caption=Run development server, label=lst:run_server]
./manage.py runserver
\end{lstlisting}
\section{Models}

\subsection{Course model}
Within models information is stored for the web application. We use one model witch stores a course and the number of votes it got. A positive vote increases the vote counter. A negative vote decreases it. Create the file \emph{scorecard/course.py} and fill it with the following content:

\begin{lstlisting}[style=Python, caption=scorecard/models.py, label=lst:models.py]
from django.db import models


class Course(models.Model):
    """
    This class stores the information about one course
    course_title    is the field which stores the title of a course
    vote            is the number of votes a course got
    pk              is created automatically as the primary key
    """
    course_title = models.CharField(max_length=200)
    votes = models.IntegerField(default=0)
\end{lstlisting}

As you can see, there is one character field with maximum length of 200 characters, that stores teh title of a course. Additionally there is a field votes that stores the number of votes one course got. The number of votes can go negative. Now the model has to be registered with our django installation. Run:
\begin{lstlisting}[style=Bash, caption=Register models, label=lst:register_models]
./manage.py makemigrations scorecard
./manage.py migrate
\end{lstlisting}

\subsection{Admin app for editing models}
If you try to access your django application (\url{http://127.0.0.1:8000/scorecard}, you remember?), you will get an error message. We have not yet defined the framework to show anything. To see if your model was propperly created and interact with it, django provides an admin interface in its package django.contrib.admin. This one is activated by default if you create a new django application.

To be shown on the admin interface, the new model needs to be enabled. You can do this by adding the model to the file \emph{scorecard/admin.py}.
\begin{lstlisting}[style=Python, caption=scorecard/admin.py, label=lst:admin.py]
from django.contrib import admin
from scorecard.models import Course

admin.site.register(Course)
\end{lstlisting}


To access it, you need to create a user first (In the version from github, the credentials are admin:admin). Run:
\begin{lstlisting}[style=Bash, caption=Create superuser, label=lst:create_superuser]
./manage.py createsuperuser
\end{lstlisting}
Now you can access \url{http://127.0.0.1:8000/admin} and create some courses as you like. As you can see, the course overview is not named nicely. This is because django does not know how to convert the course objects to a string. Add a \lstinline|__str__| method to the model to solve this. We let the string conversion return the title of the course, as this is human readable and for our case identifies the courses.

\begin{lstlisting}[style=Python, caption=scorecard/models.py, label=lst:models.py1]
from django.db import models


class Course(models.Model):
    """
    This class stores the information about one course
    course_title    is the field which stores the title of a course
    vote            is the number of votes a course got
    pk              is created automatically as the primary key
    """
    course_title = models.CharField(max_length=200)
    votes = models.IntegerField(default=0)

    def __str__(self):
        """
        Convert a Course Model to a human readable string
        :return: Returns the title of the course
        """
        return self.course_title
\end{lstlisting}

The admin application can be highly modified to match your needs concerning ordering or search of model content. However this is not covered in this tutorial. Please refer to the official documentation for more information.
\section{Views}

If you read further, you might notice that what is called views in django as mostly called controllers. Do not get bothered by that. It is correct. What is usually called view is named templates in django \footnote{\url{https://docs.djangoproject.com/en/1.7/faq/general/\#django-appears-to-be-a-mvc-framework-but-you-call-the-controller-the-view-and-the-view-the-template-how-come-you-don-t-use-the-standard-names}}. This is just a naming convention of django and has no effect on our code.

\subsection{Base views}
To make django return something when accessed using the browser, we need views. Put the following into \emph{scorecard/views.py}. This creates a view for the index, where a static string is returned. For the votes page the view shows a string that contains some variables it got delivered. These views are only here to show a basic concept of views. We will put there some real content later on. The request variable is given to the views and identifies the request. As a return value you will send a HttpResponse.

\begin{lstlisting}[style=Python, caption=scorecard/views.py, label=lst:views.py]
from django.shortcuts import render
from django.http import HttpResponse


def index(request):
    return HttpResponse("This is the index page.")


def vote(request, pk, vote):
    return HttpResponse("You are voting on %s with %s." % (pk, vote))
\end{lstlisting}

\subsection{URL redirects}
We now need to make sure, that the corresponding URLs are given to the appropriate views. Django takes the regular expressions defined in the \emph{urls.py} files and gives those to the appropriate views. We create a file \emph{scorecard/urls.py}, register the URLs in there and register this file in the file \emph{webtech/urls.py}. This is because of the application separation that is part of django. Calls that go to \lstinline|/scorecard/| are identified by the main url configuration and given to the one for our application. If there is nothing more in the url (identified by \lstinline|^$|), the index view is called. If there url has an arbitrary number of digits and then a 1 or -1, the call is given to the vote function. The additional URL parts are given to the voting as variables pk and votes. This is identified by the regular expression \lstinline|^(?P<pk>\d+)/(?P<vote>-?1)/$|.
\begin{lstlisting}[style=Python, caption=scorecard/urls.py, label=lst:scorecard_urls.py]
from django.conf.urls import patterns, url

from scorecard import views

urlpatterns = patterns('',
                       url(r'^$', views.index, name='index'),
                       url(r'^(?P<pk>\d+)/(?P<vote>-?1)/$', views.vote, name='vote'),
                       )
\end{lstlisting}

\begin{lstlisting}[style=Python, caption=webtech/urls.py, label=lst:webtech_urls.py]
from django.conf.urls import patterns, include, url
from django.contrib import admin

urlpatterns = patterns('',
                       url(r'^admin/', include(admin.site.urls)),
                       url(r'^scorecard/', include('scorecard.urls', namespace="scorecard")),
)
\end{lstlisting}

You can now access pages of the following format:
\begin{itemize}
 \item \url{http://127.0.0.1:8000/scorecard/}
 \item \url{http://127.0.0.1:8000/scorecard/1/1}
 \item \url{http://127.0.0.1:8000/scorecard/1/-1}
\end{itemize}

You will see, that these views are neither connected to your models nor look nicely. The next step is to create some templates to polish up the view and implement the business logic within the views.
\section{Templates}

Now let's beautify our websites a bit. We will take care of the functionality later. In the directory \emph{scorecard/templates/scorecard} there is a default template already. We will fill it up with some content. In \emph{scorecard/static/scorecard} there is a place for static files. I have put a css file and some logos there already.

\subsection{Index template}

In the index template we can refer to template commands such as urls, static files or variables with \lstinline||. Command structures such as if statements or loops are set in \lstinline|{{ }}|. Within the index file, we first check, if there is any object\_list given. If not, this means, that there is no course stored in the model and we need to show an error. If there are courses given, we iterate over the courses and show the course name with its votes and links to vote the course up or down. \lstinline|{{ course }}| relates to the course title, because of the \lstinline|__str__| function, we defined earlier. With \lstinline|url| you can build urls based on the names defined in the URL configuration files. \lstinline|scorecard:vote| means, that the url shall be build from the URL with the name vote within the namespace scorecard. Create the file \emph{scorecard/templates/scorecard/index.html} with the content:

\begin{lstlisting}[style=HTML, caption=scorecard/templates/scorecard/index.html, label=lst:index.html]
<!-- Make the connection to djangos static file system -->

<!DOCTYPE html>
<html>
<head lang="en">
    <meta charset="UTF-8">
    <title>Scorecard</title>
    <!-- Load stylesheets as static files -->
    <link rel="stylesheet" type="text/css" href="" />
    <link rel="stylesheet" type="text/css" href="" />
</head>
    <div class="page-header"><h1>Scorecard <small>A <a href="https://wwwmatthes.in.tum.de/pages/g78qcvnwz3u3/Web-Technologies-Frameworks-Libraries-and-Plattforms" target="_blank">webtech</a> page by <a href="http://phynformatik.de/" target="_blank">Janosch Maier</a></small> <span id="tumlogo">TUM</span><span id="inlogo">in.tum</span></h1></div>
    <!-- Check if there are any courses -->
    
    <ul>
    <!-- Iterate over the courses -->
    
        <!-- List a course and the voting possibilities -->
        <li><strong>{{ course }}</strong> has {{ course.votes }} vote{{ course.votes|pluralize }}
            (<a href=""><span class="glyphicon glyphicon glyphicon-thumbs-up" aria-hidden="true"></span></a>
            <a href=""><span class="glyphicon glyphicon glyphicon-thumbs-down" aria-hidden="true"></span></a>)
        </li>
    
    </ul>
    
        <!-- Show an error if no course existing -->
        <div class="alert alert-info" role="alert">No courses available</div>
    
</body>
</html>
\end{lstlisting}

\subsection{Updated view}

To make use of the template file, we need to update the view. The generic views classes allow the programmer to use predefined classes for recurring patterns such as lists of elements or detail views. We just give the view class the name of its template file and the model to work with. Additionally we order the list of courses by the number of votes.

\begin{lstlisting}[style=Python, caption=views.py, label=lst:views.py1]
from django.shortcuts import render
from django.http import HttpResponse
from django.views import generic

from scorecard.models import Course

class IndexView(generic.ListView):
    """
    The index page is a generic ListView.
    All Courses are shown in a list
    """
    template_name = 'scorecard/index.html'  # Template to use
    model = Course                          # Model to use

    def get_queryset(self):
        """
        Order the Ccurses by their votes begining from the course with most votes.
        """
        return Course.objects.get_queryset().order_by('-votes')

def vote(request, pk, vote):
    return HttpResponse("You are voting on %s with %s." % (pk, vote))
\end{lstlisting}

\subsection{Updated URLs}

The new view class need to be called, when the index page is requested. Therefore we need to change the \emph{scorecard/urls.py} file.

\begin{lstlisting}[style=Python, caption=scorecard/urls.py, label=lst:scorecard_urls.py1]
from django.conf.urls import patterns, url

from scorecard import views

urlpatterns = patterns('',
                       url(r'^$', views.IndexView.as_view(), name='index'),
                       url(r'^(?P<pk>\d+)/(?P<vote>-?1)/$', views.vote, name='vote'),
                       )
\end{lstlisting}


\section{Business Logic}

What is now still missing is the voting function. We already have defined the votes view in the urls files. If such a link is clicked, we will check its input values and update the votes counter. Afterwards the view redirects the user back to the index file.

\begin{lstlisting}[style=Python, caption=scorecard/views.py, label=lst:views.py2]
from django.shortcuts import get_object_or_404
from django.http import HttpResponseRedirect, Http404
from django.views import generic
from django.core.urlresolvers import reverse

from scorecard.models import Course

class IndexView(generic.ListView):
    """
    The index page is a generic ListView.
    All Courses are shown in a list
    """
    template_name = 'scorecard/index.html'  # Template to use
    model = Course                          # Model to use

    def get_queryset(self):
        """
        Order the Ccurses by their votes begining from the course with most votes.
        """
        return Course.objects.get_queryset().order_by('-votes')

def vote(request, pk, vote):
    """
    You can either up- or down-vote a course
    The vote is saved in the model
    Then the user is redirected to the index page
    :param request: Request to work on
    :param pk:      Primary key of the Course
    :param vote:    1 or -1 depending on up/down-vote
    :return:        Redirect to Index view or 404 error
    """
    # If there is no course with the corresponding pk return an error
    course = get_object_or_404(Course, pk=pk)
    if(vote == '1'):
        # Increase Vote
        course.votes += 1
        course.save()
        return HttpResponseRedirect(reverse('scorecard:index'))
    elif(vote == '-1'):
        # Decrease Vote
        course.votes -= 1
        course.save()
        return HttpResponseRedirect(reverse('scorecard:index'))
    else:
        # Vote invalid
        return Http404('Vote must be either 1 or -1')
\end{lstlisting}

Now the voting system is fully functional.
\section{Tests}

Django has a sophisticated testing framework. You can directly test your functions with unit tests and do also further tests that simulate requests. The file \emph{scorecard/tests.py} contains the tests for the scorecard application. Test-driven development and continuous integration are posible with django.

\subsection{Configure the Testing Environment}
The class ScoreboardTest contains the tests for the scoreboard application. When tests are run, the application creates an extra database for the tests. all data in your productive environment is not touched. The \lstinline|setUp()| method (line 14) creates objects for the test cases. It is run automatically by the testing environment before each test. The test methods will use the defined lecturers and courses to run their tests.

\begin{lstlisting}[style=Python, caption=Exceprt from scorecard/tests.py, label=lst:tests.py]
from django.test import TestCase
from django.core.urlresolvers import reverse

from scorecard.models import Course, Lecturer


class ScoreboardTest(TestCase):
    lecturer_1 = Lecturer()
    lecturer_2 = Lecturer()
    course_1 = Course()
    course_2 = Course()
    course_3 = Course()

    def setUp(self):
        """
        Create some courses & lecturers
        """
        self.lecturer_1 = Lecturer.objects.create(first_name="Janosch", last_name="Maier")
        self.lecturer_2 = Lecturer.objects.create(first_name="A", last_name="B")
        self.course_1 = Course.objects.create(course_title="EADS", votes=0, lecturer=self.lecturer_1)
        self.course_2 = Course.objects.create(course_title="ITSec", votes=0, lecturer=self.lecturer_2)
        self.course_3 = Course.objects.create(course_title="Webtech", votes=0, lecturer=self.lecturer_2)
\end{lstlisting}

The test functions use several other methods for convenience. Important is \lstinline|vote()| (line 10), which takes a course and either 1 or -1 and votes the course up and down. The function returns the votes before and after the vote. This is not done directly within the model but using the views. \lstinline|self.client.get()| runs a get request on a page of the project. \lstinline|reverse| does a lookup in the \emph{urls.py} file to get the correct URL. \lstinline|vote_again()| (line 24) calls the vote\_again page to reenable the voting if a user in this session has already voted. This functionality is covered in section \ref{ssec:session} of this tutorial.

\begin{lstlisting}[style=Python, caption=Exceprt from scorecard/tests.py, label=lst:tests.py]
    def delete_courses(self):
        """
        Delete all courses for test
        :return:
        """
        self.course_1.delete()
        self.course_2.delete()
        self.course_3.delete()

    def vote(self, course, up_down):
        """
        Vote for course with vote up_down
        :param course: Course object
        :param up_down: 1 or -1
        :return: old_votes, new_votes
        """
        response = self.client.get(reverse('scorecard:index'))
        old_votes = response.context['object_list'].get(pk=course.pk).votes
        self.client.get(reverse('scorecard:vote', kwargs={'pk': course.pk, 'vote': up_down}))
        response = self.client.get(reverse('scorecard:index'))
        new_votes = response.context['object_list'].get(pk=course.pk).votes
        return old_votes, new_votes

    def vote_again(self):
        """
        Reset has_already_voted hook
        :return:
        """
        self.client.get(reverse('scorecard:vote_again'))
\end{lstlisting}

\subsection{Unittests}
\lstinline|test_lecturer_str()| works directly on the model and checks if the \lstinline|__str__()| method of the lecturer model works correctly

\begin{lstlisting}[style=Python, caption=Exceprt from scorecard/tests.py, label=lst:tests.py]
    def test_lecturer_str(self):
        """
        Test if the __str__ method of lecturer works properly
        :return:
        """
        lecturer = Lecturer.objects.create(first_name='Janosch', last_name='Maier')
        self.assertEqual(lecturer.__str__(), 'Janosch Maier')
\end{lstlisting}


\lstinline|test_great_course_for_bad_course()| and \lstinline|test_great_course_for_great_course()| also work on the model directly. They use the function \lstinline|is_great_course()|, which is defined in the course model. Those are regular unit tests.

\begin{lstlisting}[style=Python, caption=Exceprt from scorecard/tests.py, label=lst:tests.py]
    def test_great_course_for_bad_course(self):
        """
        Return false if course has less then 100 votes
        :return:
        """
        course = Course.objects.create(course_title="EADS", votes=100, lecturer=self.lecturer_1)
        self.assertEqual(course.is_great_course(), False)

    def test_great_course_for_great_course(self):
        """
        Return false if course has less then 100 votes
        :return:
        """
        course = Course.objects.create(course_title="EADS", votes=101, lecturer=self.lecturer_1)
        self.assertEqual(course.is_great_course(), True)
\end{lstlisting}

\subsection{Testing Webserver Respnoses}
\lstinline|test_error_msg_if_no_course()| first deletes all courses and then checks if the response for a get request on the index page contains the error message, that there are no courses available. This check is done using the \lstinline|assertContains()| method.

\begin{lstlisting}[style=Python, caption=Exceprt from scorecard/tests.py, label=lst:tests.py]
    def test_error_msg_if_no_course(self):
        """
        If there is no Course existing, the Index should present an error msg
        """
        self.delete_courses()
        response = self.client.get(reverse('scorecard:index'))
        self.assertContains(response, 'No courses available')
\end{lstlisting}


\lstinline|test_if_courses_are_shown_without_voting()| checks if all courses created in the \lstinline|setUp()| method are given to the index template correctly. \lstinline|assertQuerysetEqual()| checks if the objext\_list in the context equals the one that contains all the defined courses for the test case.

\begin{lstlisting}[style=Python, caption=Exceprt from scorecard/tests.py, label=lst:tests.py]
    def test_if_courses_are_shown_without_voting(self):
        """
        If courses are created, they should be shown on the index page
        """
        response = self.client.get(reverse('scorecard:index'))
        self.assertQuerysetEqual(response.context['object_list'], ['<Course: EADS>', '<Course: ITSec>', '<Course: Webtech>'])
\end{lstlisting}

\lstinline|test_if_voting_redirect_is_correct()| looks if the status code of the response when the voting page is called is a correct redirect.

\begin{lstlisting}[style=Python, caption=Exceprt from scorecard/tests.py, label=lst:tests.py]
    def test_if_voting_redirect_is_correct(self):
        """
        If a vote is counted, the user is redirected
        """
        response = self.client.get(reverse('scorecard:vote', kwargs={'pk':1, 'vote':1}))
        self.assertEqual(response.status_code, 302)
\end{lstlisting}

The following 4 test methods test the voting functionality. A vote shall only be counted when the user in the session has not yet voted (again, see section \ref{ssec:session} for more details). They use the \lstinline|vote()| method to call the voting page and compare the votes before and afterwards.

\begin{lstlisting}[style=Python, caption=Exceprt from scorecard/tests.py, label=lst:tests.py]
    def test_if_voting_increased_counter(self):
        """
        If a positive vote is counted, the counter should increase
        """
        vote = 1
        old_votes, new_votes = self.vote(self.course_1, vote)
        self.assertEqual(old_votes + vote, new_votes)

    def test_if_voting_decreased_counter(self):
        """
        If a negative vote is counted, the counter should decrease
        """
        vote = -1
        old_votes, new_votes = self.vote(self.course_1, vote)
        self.assertEqual(old_votes + vote, new_votes)

    def test_if_voting_increased_counter_twice_without_reset(self):
        """
        If a positive vote is counted, the counter should increase
        """
        vote = 1
        old_votes, not_used = self.vote(self.course_1, vote)
        not_used, new_votes = self.vote(self.course_1, vote)
        self.assertEqual(old_votes + vote, new_votes)

    def test_if_voting_increased_counter_twice_with_reset(self):
        """
        If a positive vote is counted, the counter should increase
        """
        vote = 1
        old_votes, not_used = self.vote(self.course_1, vote)
        self.vote_again()
        not_used, new_votes = self.vote(self.course_1, vote)
        self.assertEqual(old_votes + (2 * vote), new_votes)
\end{lstlisting}

\lstinline|test_statistic()| checks if the statistic pages contains the correct values that should be calculated with the lecturer and course objects as created using \lstinline|setUp()|. lecturer\_1 has only one course that is voted with two positive votes. The other lecture has an average of zero votes. Therefore lecturer\_1 is the best lecturer with a votes mean of two.

\begin{lstlisting}[style=Python, caption=Exceprt from scorecard/tests.py, label=lst:tests.py]
    def test_statistic(self):
        """
        Statistic should show correct values
        :return:
        """
        vote = 1
        self.vote(self.course_1, vote)
        self.vote_again()
        self.vote(self.course_1, vote)
        response = self.client.get(reverse('scorecard:statistics'))
        self.assertEqual(response.context['lecturer_best'], self.lecturer_1)
        self.assertEqual(response.context['lecturer_best_votes_mean'], 2)
\end{lstlisting}

To run the tests, use the following command:
\begin{lstlisting}[style=Bash, caption=Run tests, label=lst:run_tests]
./manage.py tests
\end{lstlisting}
\section{Messages}

With django you can easily create messages and propagate them in between your pages. The messages application is already installed, if you create a new project. You just need to put some few statements into your \emph{webtech/config.py} to make it work. The second statement is used to make error messages work propperly with bootstrap. The css class therefore has to be set to danger, not error.
\begin{lstlisting}[style=Python, caption=webtech/config.py, label=lst:config.py1]
TEMPLATE_CONTEXT_PROCESSORS = (
    'django.contrib.messages.context_processors.messages',
    'django.contrib.auth.context_processors.auth'
)

from django.contrib.messages import constants as messages
MESSAGE_TAGS = {
    messages.ERROR: 'danger'
}
\end{lstlisting}

Update your \emph{views.py} in the following way to add some messages.
\begin{lstlisting}[style=Python, caption=Add messages to views, label=lst:views_msg]
def updateVote(course, vote):
    if vote == '1':
        course.votes += 1
        course.save()
        return  True
    elif vote == '-1':
        course.votes -= 1
        course.save()
        return True
    return False


def vote(request, pk, vote):
    """
    You can either up- or down-vote a course
    The vote is saved in the model
    Then the user is redirected to the index page
    :param request: Request to work on
    :param pk:      Primary key of the Course
    :param vote:    1 or -1 depending on up/down-vote
    :return:        Redirect to Index view or 404 error
    """
    # If there is no course with the corresponding pk return an error
    course = get_object_or_404(Course, pk=pk)
    if updateVote(course):
        messages.add_message(request, messages.SUCCESS, "Vote Successful")
        return HttpResponseRedirect(reverse('scorecard:index'))
    else:
        # Vote invalid
        messages.add_message(request, messages.WARNING, "Vote Not Successful")
        return HttpResponseRedirect(reverse('scorecard:index'))
\end{lstlisting}

To show the messages on your views page, use the following code. This uses djangos message.tags attribute to get the alert class for bootstrap.
\begin{lstlisting}[style=HTML, caption=index.html, label=lst:index.html1]
    <!-- Show message if stored -->
    
        
            <div class="alert alert-{{ message.tags }}">{{ message }}</div>
        
    
\end{lstlisting}
\section{Session}

You need to add the request context\_processor, so you can access the request.session variable in the view/template
\begin{lstlisting}[style=Python, caption=webtech/config.py, label=lst:config.py1]
TEMPLATE_CONTEXT_PROCESSORS = (
    'django.contrib.messages.context_processors.messages',
    'django.contrib.auth.context_processors.auth',
    'django.core.context_processors.request'
)
\end{lstlisting}

Update your \emph{views.py} in the following way to add the session storage handling.
\begin{lstlisting}[style=Python, caption=Add messages to views, label=lst:views_msg]def vote(request, pk, vote):
    """
    You can either up- or down-vote a course
    The vote is saved in the model
    Then the user is redirected to the index page
    :param request: Request to work on
    :param pk:      Primary key of the Course
    :param vote:    1 or -1 depending on up/down-vote
    :return:        Redirect to Index view or 404 error
    """
    # If there is no course with the corresponding pk return an error
    course = get_object_or_404(Course, pk=pk)
    if request.session.get('has_voted', False):
        messages.add_message(request, messages.ERROR, "You have already voted!")
        return index()
    if updateVote(course, vote):
        request.session['has_voted'] = True
        messages.add_message(request, messages.SUCCESS, "Vote Successful!")
        return index()
    else:
        # Vote invalid
        messages.add_message(request, messages.ERROR, "Vote Not Successful!")
        return index()

def vote_again(request):
    request.session['has_voted'] = False
    return index()
\end{lstlisting}

To show the messages on your views page, use the following code. This uses djangos message.tags attribute to get the alert class for bootstrap.
\begin{lstlisting}[style=HTML, caption=index.html, label=lst:index.html1]
    
        <div class="alert alert-info">
        You usually can only vote once. If you want to vote again, I can make an exception for you. <a href="">Vote Again?</a>
        </div>
    
\end{lstlisting}

Add a new URL redirect so to make everything work.
To show the messages on your views page, use the following code. This uses djangos message.tags attribute to get the alert class for bootstrap.
\begin{lstlisting}[style=Python, caption=urls.py, label=lst:urls.py1]
urlpatterns = patterns('',
                       url(r'^$', views.IndexView.as_view(), name='index'),
                       url(r'^(?P<pk>\d+)/(?P<vote>-?\d)/$', views.vote, name='vote'),
                       url(r'^vote_again$', views.vote_again, name='vote_again')
                       )
\end{lstlisting}
\section{Translation}

To make your django application available for a broad audience, internationalization is needed. With the translation framework of django this is possible.

\subsection{Translation definitions}
To create an internationalized version of the scorecard application, all strings need to put into translation functions. Based on these translation strings, the \emph{django-admin.py} file creates a \emph{.po} file, in which the translation takes place. First in your views (or wherever strings are created), import the translation method enclose the strings accordingly. Look at the string in line 6.

\begin{lstlisting}[style=Python, caption=exceprt of views.py with translation, label=lst:views.py_translation]
from django.utils.translation import ugettext as _

def vote(request, pk, vote):
    course = get_object_or_404(Course, pk=pk)
    if has_already_voted(request):
        messages.add_message(request, messages.ERROR, _("You have already voted!"))
\end{lstlisting}

For translating templates, the \lstinline|| tag is available. See how translation for the menu links is initialized in lines 5 and 6 of the following snippet.

\begin{lstlisting}[style=HTML, caption=exceprt of base.html with translation, label=lst:base.html_translation]


<div class="navbar-collapse" id="bs-example-navbar-collapse-1">
    <ul class="nav navbar-nav">
        <li><a href=""></a></li>
        <li><a href=""></a></li>
    </ul>
</div><!-- /.navbar-collapse -->
\end{lstlisting}

\subsection{Pluralization}
Pluralization is a bit more difficult. If English is set as the first language, the plural ``s'' can be used easily with the \lstinline|pluralize| command. This is not possible for more complicated languages such as German or any than the first language used in the project. To create a pluralized version of a string, you need a counter that defines whether the singular or plural version of the string is used. In the following code, \lstinline|course.votes| is used as counter, to pluralize the votes.

\begin{lstlisting}[style=HTML, caption=exceprt of index.html with translation, label=lst:index.html_translation]
{{ counter }} vote.{{ counter }} votes.
\end{lstlisting}

\subsection{Translation Files}
Now we defined the strings that shall be translated. To do the actual translation, let django create a translation file. Create the folder \emph{scorecard/locale}.

From within the \emph{scorecard} directory run the following command to create a translation file for German:

\begin{lstlisting}[style=Bash, caption=Create a German translation file, label=lst:makemessages]
django-admin.py makemessages -l de
\end{lstlisting}

Now in the directory \emph{scorecard/locale}, a directory for the German language appears. Deep down in this directory there is a \emph{django.po} file. Create the translation strings within this file. You can see the plural version of the string for the votes counter in the \emph{index.html} file in line 9-14.

\begin{lstlisting}[style=HTML, caption=django.po file for German, label=lst:django.po]
#: templates/scorecard/base.html:20
msgid "Home"
msgstr "Start"

#: templates/scorecard/base.html:21
msgid "Statistics"
msgstr "Statistik"

#: templates/scorecard/index.html:10
#, python-format
msgid "%(counter)s vote."
msgid_plural "%(counter)s votes."
msgstr[0] "%(counter)s Stimme."
msgstr[1] "%(counter)s Stimmen."
\end{lstlisting}

To apply the translated messages, run:
\begin{lstlisting}[style=Bash, caption=Compile messages, label=lst:compilemessages]
django-admin.py compilemessages
\end{lstlisting}
\end{document}
