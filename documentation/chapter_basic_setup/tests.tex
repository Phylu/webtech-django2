\section{Tests}

Django has a sophisticated testing framework. You can directly test your functions (unit tests) and do also further tests that simulate requests. To create some basic tests for the scorecard application put the following into \emph{scorecard/tests.py}. Test-driven development and continuous integration are posible with django. The create courses function acts directly on the model and creates some courses. Those are only valid within the test case function that calls it. Otherwise a test database is used and all data stored in the developing environment are not touched by those tests. The client get methods simulaties request on the webpage and the reverse function is used to create the correct URL based on the view to be called and the parameters given. Assertion is done like in regular unit tests.

\begin{lstlisting}[style=Python, caption=scorecard/tests.py, label=lst:tests.py]
from django.test import TestCase
from django.core.urlresolvers import reverse

from scorecard.models import Course

def create_courses():
    """
    Create some courses
    """
    Course.objects.create(course_title="EADS", votes=0)
    Course.objects.create(course_title="ITSec", votes=0)
    Course.objects.create(course_title="Webtech", votes=0)

class ScoreboardTest(TestCase):

    def test_if_error_msg_if_no_course(self):
        """
        If there is no Course existing, the Index should present an error msg
        """
        response = self.client.get(reverse('scorecard:index'))
        self.assertContains(response, 'No courses available')

    def test_if_courses_are_shown_without_voting(self):
        """
        If courses are created, they should be shown on the index page
        """
        create_courses()
        response = self.client.get(reverse('scorecard:index'))
        self.assertQuerysetEqual(response.context['object_list'], ['<Course: EADS>', '<Course: ITSec>', '<Course: Webtech>'])

    def test_if_voting_redirect_is_correct(self):
        """
        If a vote is counted, the user is redirected
        """
        create_courses()
        response = self.client.get(reverse('scorecard:vote', kwargs={'pk':1, 'vote':1}))
        self.assertEqual(response.status_code, 302)

    def test_if_voting_increased_counter(self):
        """
        If a positive vote is counted, the counter should increase
        """
        create_courses()
        response = self.client.get(reverse('scorecard:index'))
        old_votes = response.context['object_list'].get(pk=1).votes
        response = self.client.get(reverse('scorecard:vote', kwargs={'pk':1, 'vote':1}))
        response = self.client.get(reverse('scorecard:index'))
        new_votes = response.context['object_list'].get(pk=1).votes
        self.assertEqual(old_votes + 1, new_votes)

    def test_if_voting_decreased_counter(self):
        """
        If a negative vote is counted, the counter should decrease
        """
        create_courses()
        response = self.client.get(reverse('scorecard:index'))
        old_votes = response.context['object_list'].get(pk=1).votes
        response = self.client.get(reverse('scorecard:vote', kwargs={'pk':1, 'vote':-1}))
        response = self.client.get(reverse('scorecard:index'))
        new_votes = response.context['object_list'].get(pk=1).votes
        self.assertEqual(old_votes - 1, new_votes)
\end{lstlisting}

To run the tests, use the following command:
\begin{lstlisting}[style=Bash, caption=Run tests, label=lst:run_tests]
./manage.py tests
\end{lstlisting}